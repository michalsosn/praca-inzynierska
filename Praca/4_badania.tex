\chapter{Usprawnienia Glasgow Haskell Compiler}\label{chap:badania}
\todo[inline,size=\tiny]{Ten rozdział zawiera opis wyników uzyskanych w~ramach pracy. Jeśli praca miała
cel badawczy należy skupić się na opisie przeprowadzonych eksperymentów oraz
prezentacji i~analizie uzyskanych wyników. Jeśli praca nie miała na celu
uzyskania nowatorskich wyników, należy skupić się na opisie architektury
stworzonej aplikacji. W~obu przypadkach podstawowym celem tego rozdziału jest
realizacja celów postawionych w~rozdziale \ref{sec:cele_pracy}. Rozdział ten ma
bezspornie pokazywać, że cele pracy zostały zrealizowane}

\sectionex{Zgłoszenie nr 10839}{Consistent pretty-printing of type families}
\todo[inline]{Treść 1}

\sectionex{Zgłoszenie nr 10982}{Warn about unused pattern variables in type families}
\todo[inline]{Treść 2}

\sectionex{Zgłoszenie nr 11098}{PartialTypeSignatures mishandles type variables that begin with an underscore}
\todo[inline]{Treść 3}
