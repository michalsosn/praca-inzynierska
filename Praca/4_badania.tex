\chapter{Usprawnienia Glasgow Haskell Compiler}\label{chap:badania}
\todo[disable,inline,size=\tiny]{Ten rozdział zawiera opis wyników uzyskanych w~ramach pracy. Jeśli praca miała
cel badawczy należy skupić się na opisie przeprowadzonych eksperymentów oraz
prezentacji i~analizie uzyskanych wyników. Jeśli praca nie miała na celu
uzyskania nowatorskich wyników, należy skupić się na opisie architektury
stworzonej aplikacji. W~obu przypadkach podstawowym celem tego rozdziału jest
realizacja celów postawionych w~rozdziale \ref{sec:cele_pracy}. Rozdział ten ma
bezspornie pokazywać, że cele pracy zostały zrealizowane}

\sectionex{Zgłoszenie nr 10839}{Consistent pretty-printing of type families}

\subsection{Wymagania}
\todo[inline]{Treść 11}

\begin{lstlisting}[caption={Listing test}]
type family F a b
type instance F a b = a

fun :: Int -> Int
fun a = 10 + 1

-- NamedWildcardExplicitForall.hs:14:16: error:
--    • Couldn't match expected type ‘Bool’ with actual type ‘_a’

for (i = 0, i < 10; ++i) { a += 10; }
\end{lstlisting}


\subsection{Rozwiązanie}
\todo[inline]{Treść 12}

\subsection{Testy}
\todo[inline]{Treść 13}

\sectionex{Zgłoszenie nr 10982}{Warn about unused pattern variables in type families}

\subsection{Wymagania}
\todo[inline]{Treść 21}

\subsection{Rozwiązanie}
\todo[inline]{Treść 22}

\subsection{Testy}
\todo[inline]{Treść 23}

\sectionex{Zgłoszenie nr 11098}{PartialTypeSignatures mishandles type variables that begin with an underscore}

\subsection{Wymagania}
\todo[inline]{Treść 31}

\subsection{Rozwiązanie}
\todo[inline]{Treść 32}

\subsection{Testy}
\todo[inline]{Treść 33}
