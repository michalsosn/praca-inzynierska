\chapter{Użyte narzędzia i~technologie}\label{chap:technologie}

\section{Język programowania}\label{sec:jezyk_programowania}

Glasgow Haskell Compiler został napisany w~języku Haskell. Wykorzystano została
w~projekcie technika \foreign{bootstraping}, polegająca na tym, iż kompilator
rozwijany jest w~języku, który kompiluje i~jedną z~zależności potrzebną do
zbudowania GHC jest on sam. Środowiskiem wykorzystanym do edycji kodu było
IntelliJ IDEA z~wtyczką zapewniającą wsparcie dla języka Haskell. Wtyczka ta
niestety w~pełni działa tylko z~projektami systemu Cabal i~nie jest dostosowana
do prac nad GHC. Z~tego powodu środowisko nie zapewniało na przykład ciągłego
raportowania błędów w~kodzie lub wyszukiwania deklaracji określonego typu, choć
z~tych udogodnień można korzystać w~innych projektach.

Testowanie i~walidacja kompilatora są realizowane przez skrypty w~języku
Python. Jest on wymagany do zbudowania GHC
również z~tego powodu, że do stworzenia dokumentacji wykorzystane zostało
narzędzie Sphinx, wymagające interpretera.

W~projekcie użyty został również język C do napisania środowiska
uruchomieniowego, linkowanego z~każdym programem skompilowanym przez GHC.
W~pozostałych częściach GHC można znaleźć dyrektywy preprocesora C, który GHC może
wykorzystać zanim przejdzie do kompilacji. W~tej pracy jednak kontakt z~nim był
znikomy.

\section{Narzędzia}

\subsection{Git}

Kod GHC znajduje się w~repozytorium Git i~jest publicznie dostępny do
pobrania pod adresem \url{http://git.haskell.org/ghc.git}.
Biblioteki wykorzystywane w~kompilatorze również są przechowywane
w~repozytoriach Gita, lecz są osobnymi projektami, dołączonymi do GHC jako
submoduły. Klon oficjalnego repozytorium znajduje się również w~popularnym
serwisie GitHub. Prawa zapisu do repozytorium są
ograniczone\cite{WikiGettingTheSources}.

Krokiem wymaganym do rozpoczęcia pracy było sklonowanie repozytorium i~lokalne
zbudowanie kompilatora ze źródeł. Po lokalnej edycji plików zmiany musiały
zostać zapisane w~repozytorium i~opatrzone komentarzem stosującym się do
konwencji opisanej na wiki GHC, zanim mogły zostać wysłane z~użyciem systemu
Phabricator.

\subsection{Trac}

Trac to system do organizacji pracy programistów. Przechowuje listę zgłoszeń,
czyli spójnych propozycji zmian w~projekcie. Zapisywane są w~nim zadania do
wykonania oraz błędy wykryte w~GHC, także przez zwykłych użytkowników. Chroni to
je przed zapomnieniem w~razie, gdyby nie było możliwe zajęcie się nimi od
razu. Znajdują się w~tym systemie również propozycje dodania nowych funkcji. Do
każdego zgłoszenia może być dołączony opis, w~przypadku błędu sposób
reprodukcji, zapewnione jest miejsce na dyskusję i~mechanizm śledzenia stanu
wykonania zadania. Aby ułatwić organizację pracy, możliwe jest przypisanie do zgłoszeń
właściciela, by zapobiec sytuacji, w~której tę samą rzecz wykonuje
wiele osób. Możliwe jest określenie priorytetu i~planowanej wersji, w~której
zadanie powinno już być wykonane i~późniejsze sprawdzanie, czy ten cel został
osiągnięty. Ułatwione jest również umieszczanie odnośników do innych,
powiązanych zgłoszeń czy do systemu Phabricator. Wszystkie te funkcje
wykorzystywane są w~GHC.

\begin{figure}[ht]
    \centering
    \includegraphics[width=0.8\textwidth]{images/Trac_description}
    \caption{Opis zgłoszenia w systemie Trac}
    \label{fig:Trac_description}
\end{figure}

Przykład zgłoszenia widoczny jest na rysunku \ref{fig:Trac_description}. Dotyczy
ono jednego z~wykonanych w~pracy usprawnień, pozostałe również mają swoje opisy
w~systemie. Aktualizacja informacji na stronie Trac jest wymagana przez
procedurę wprowadzania zmian w~GHC\cite{WikiFixingBugs}.

\subsection{Phabricator}

\cite{WikiPhabricator}

Phabricator wykorzystywany jest do inspekcji kodu w~czasie zgłaszania ulepszenia
oraz do automatycznej walidacji z~wykorzystaniem narzędzia
Harbormaster. Inspekcja jest ułatwiona, gdyż program prezentuje obok siebie
wersję sprzed i~po dokonaniu zmian, z~wyróżnieniem zmodyfikowanych linii
kodu. Daje też możliwość komentowania wybranych fragmentów.

\begin{figure}[ht]
    \centering
    \includegraphics[width=0.8\textwidth]{images/Phabricator_summary}
    \caption{Informacje o proponowanej zmianie w kodzie na stronie Phabricator}
    \label{fig:Phabricator_summary}
\end{figure}

Na rysunku \ref{fig:Phabricator_summary} widać opis jednego z~ulepszeń
dokonanego w~tej pracy. Zgodnie z~procedurą \cite{WikiFixingBugs}, aby zmiany
zapisane w~lokalnym repozytorium Git znalazły się na serwerze, łatka musi
przejść wpierw przez proces rewizji w~systemie Phabricator. Polega on na tym, iż
inni programiści, szczególnie ci, którzy posiadają uprawnienia do zapisu do
repozytorium, zapoznają się z~przesłanym kodem. Zgłaszają oni swoje uwagi
i~decydują, czy zmiana może zostać naniesiona na repozytorium na serwerze, czy
wymaga poprawek.

\begin{figure}[ht]
    \centering
    \includegraphics[width=0.9\textwidth]{images/Phabricator_validate}
    \caption{Wynik walidacji programu Harbormaster widoczny na stronie Phabricator}
    \label{fig:Phabricator_validate}
\end{figure}

Z~chwilą wysłania proponowanych zmian w~kodzie do Phabricatora, wykonane zostaje
również automatyczne zbudowanie i~przetestowanie nowej wersji GHC przez moduł
Harbormaster na przeznaczonym do tego serwerze. Jeżeli walidacja wykryje błędy,
jest to znak, iż konieczne są poprawki i~oszczędza to czas poświęcony na
inspekcję kodu przez programistów. Na rysunku \ref{fig:Phabricator_validate}
widać raport z~Harbormastera do jednego z~wprowadzonych w~tej pracy usprawnień,
w~którym nie wykryto problemów.

Jak wspomniane zostało w~sekcji \ref{sec:jezyk_programowania}, GHC jest budowany
z~użyciem innej wersji GHC. Aby zapewnić stabilność, istnieje wymaganie, że nowa
wersja kompilatora musi być możliwa do skompilowania z~wykorzystaniem dowolnej
z~dwóch poprzednich stabilnych wersji\cite{WikiFixingBugs}. Sam proces budowania
przebiega wieloetapowo. Najpierw, z~użyciem starszej wersji kompilatora,
budowane są biblioteki, zależności GHC. Następnie kompilowany w~ten sposób jest
sam GHC, czego rezultatem jest kompilator w~nowszej wersji. Następnie te dwa
kroki, budowania bibliotek i~kompilacji GHC, są powtarzane, lecz
z~wykorzystaniem nowego kompilatora i~to on jest uznawany za wynik całego
procesu\cite{WikiBuildSystem}. Dzięki temu sposobowi działania, usprawnienia
w~kompilacji są od razu zastosowane do kompilatora, przy jednym etapie
wykorzystywane byłyby optymalizacje i~sposób generowania kodu ze starszej
wersji. Stanowi to również dobry test nowej wersji, gdyż jeżeli pojawiły się
w~niej nowe błędy, to jest duża szansa, iż ujawnią się one w~czasie budowania
kompilatora na drugim etapie.

Uzyskany kompilator jest poddawany kilku tysiącom testów. W~większości dzielą
się one na trzy kategorie: testów, w~których kompilacja pewnego programu musi
zakończyć się powodzeniem; testów, których kompilacja musi zakończyć się błędem;
testów, w~których kompilacja musi się udać, a~następnie sprawdzane jest
zachowanie programu po uruchomieniu. Wiele testów polega na sprawdzeniu
komunikatów kompilatora i~porównaniu ze wzorcem.
