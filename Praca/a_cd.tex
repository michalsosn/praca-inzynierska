\chapter{Płyta CD}\label{app:plyta}

\begin{figure}[htb]
\makebox[\textwidth]{\framebox[12.8cm]{\rule{0pt}{12.8cm}}}
\end{figure}
\pagebreak

Zawartość katalogów na płycie:
\begin{description}
    \item[doc] : elektroniczna wersja pracy dyplomowej oraz dwie prezentacje wygłoszone podczas seminarium dyplomowego
    \item[src] : repozytorium z~kodem źródłowym aplikacji, z~gałęzią \code{master} z~dnia 2016-01-31 i~dwiema gałęziami \code{trac-10839-consistent-ppr} i~\code{trac-10982-unused-vars-in-tyfams}, na których wykonywane były prace
    \item[web] : kopie źródeł elektronicznych umieszczonych w~bibliografii
\end{description}

Przełączenie się między gałęziami w~repozytorium wymaga dodatkowego uaktualnienia submodułów używanych w~projekcie komendą \code{git submodule update --init}. Szczegóły zawarte są w~zasobie \cite{WikiGettingTheSources}, którego kopię można znaleźć w~katalogu \textbf{web}.

Zbudowanie ghc wymaga przygotowania środowiska. Opis jak tego dokonać na poszczególnych platformach jest dostępny w~\cite{WikiPreparation}. Do katalogu \textbf{web} zostały skopiowane podstrony z~instrukcjami dla systemów Windows, Linux i~MacOS X. Sam proces budowania jest opisany w~zasobie \cite{WikiNewcomers}.
