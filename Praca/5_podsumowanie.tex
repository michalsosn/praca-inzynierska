\chapter{Podsumowanie i~wnioski}\label{chap:podsumowanie}

\section{Dyskusja wyników}

Dzięki zrealizowaniu pracy, sposób wyświetlania rodzin typów w~błędach
i~ostrzeżeniach kompilatora jest ujednolicony. Dodane zostało wyświetlanie prawych
stron równań rodzin typów danych w~sposób naśladujący składnię tych
konstrukcji. Stary sposób wyświetlania, oddający wewnętrzną reprezentację
rodzin typów danych, został zachowany w~zrzucie z~etapu sprawdzania typów.

Dodane zostało także opcjonalne wyświetlanie ostrzeżeń o~nieużywanych zmiennych
w~rodzinach typów. Zmienna uważana jest za nieużywaną jeżeli występuje we
wzorcach po lewej stronie równania tylko raz i~nie występuje w~typie po prawej
stronie.

Poprawiony został błąd z~zamianą zmiennych typów zaczynających się od
podkreślnika w~symbole wieloznaczne w~kontekstach, gdzie są one
niedozwolone. Algorytm renamera został również zmieniony tak, by nie dokonywał
zamiany zmiennych jawnie związanych kwantyfikatorem. Dzięki temu uaktywnienie
rozszerzenia \code{NamedWildCards} nie powoduje już odrzucania poprawnych programów.

Wszystkie te trzy modyfikacje przeszły proces rewizji kodu w~systemie
Phabricator i~znalazły się w~repozytorium. Od tamtej pory wprowadzony kod
podlegał dodatkowym modyfikacjom i~refaktoryzacji dokonanej przez innych
programistów. W~szczególności część dotycząca \code{NamedWildCards} została
w~dużej części zastąpiona przez alternatywne rozwiązanie Simona
Peytona Jonesa. Z~kolei ostrzeżenia o~nieużywanych zmiennych typów w~wyniku
zgłoszenia nr 11451 są uaktywniane nową flagą \code{-Wunused-type-variables}
zamiast \code{-Wunused-matches}.
Jednak wszystkie wprowadzone w tej pracy usprawnienia dalej są dostępne
w~kompilatorze, dlatego cele należy uznać za zrealizowane.

\section{Perspektywy dalszych badań}
System Trac na stronie GHC zawiera obecnie ponad 1600 otwartych
zgłoszeń\cite{WikiTickets}. W~innych pracach można podjąć się dokonania następnych
usprawnień związanych z~programowaniem z~użyciem typów lub z~inną częścią
kompilatora. Złożone propozycje mają poświęcone sobie podstrony na wiki GHC,
gdzie można znaleźć ich planowane funkcje, projekty, opisy przebiegu
implementacji i~odnośniki do prac badawczych, na których bazują. Są to na
przykład propozycja dodania typów zależnych do Haskella lub wprowadzenia
definiowanych przez użytkownika błędów typów. Praca mogłaby polegać również na
sformułowaniu nowej propozycji i~zaimplementowaniu jej. Możliwości są szerokie
i~wiele osób zdecydowało się już poświęcić swój czas GHC.
