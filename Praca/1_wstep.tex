\chapter{Wstęp}\label{chap:wstep}

\js{Na początek jedna uwaga ogólna.  Najważniejszymi rozdziałami pracy są wstęp
  i podsumowanie - na pewno recenzent skupi się na nich.  Dlatego trzeba włożyć
  dużo pracy w to, aby wstęp był zrozumiały.  Jest to szczególnie prawdziwe w
  sytuacji, kiedy napisał Pan pracę na temat na którym nikt się na wydziale nie
  zna (poza mną).  Z tego powodu będę miał dużo uwag do wstępu, aby go jak
  najbardziej dopracować.}

\js{Mam kilka uwag ogólnych po przeczytaniu pierwszego rozdziału.  O ile
  przegląd literatury nie budzi moich zastrzeżeń, a cele pracy nie budzą
  większych zastrzeżeń (tzn. przydałyby się drobne korekty), o tyle sam wstęp
  jest bardzo chaotyczny i niejasny.  Ja rozumiem o co w nim chodzi, ale to
  dlatego ze znam temat.  Ktoś kto go nie zna jest bez szans.  Trzeba go będzie
  przepisać prawie od nowa.  Proszę zacząć od nakreślenia ogólnej struktury tego
  podrozdziału (np. rozpocząć od opisania czego dotyczy praca - najpierw
  ogólnie, potem bardziej szczegółowo), zaplanować jakie pojęcia trzeba wyjaśnić
  itd. itp. i mając taki ogólny zarys napisać wstęp.  W chwili obecnej porusza
  Pan mnóstwo tematów (często w jednym zdaniu), przy czym te tematy wydają się
  być bez związku z tym co było powiedziane wcześniej albo z tym co jest mówione
  później.  Jeśli wspomina Pan o typach zależnych nalezy chociaz w dwóch-trzech
  zdaniach wyjaśnić o co chodzi.  Ponadto wskazane jest wstawienie odwołań do
  literatury (praktycznie każde rozszerzenie systemu typów ma swój artykuł na
  który można się powołać - wydaje mi się że moje artykuły z 2014 i 2015 mogą
  być dobrą ściągawką) oraz linków do zasobów sieciowych (Trac GHC, Phab, itd.)}

Zakresem niniejszej pracy inżynierskiej jest informatyka, w szczególności
tworzenie kompilatorów i systemy typów.
\js{Może: ``tworzenie kompilatorów dla języków funkcyjnych ze statycznymi
  systemami typów''?}
Przedmiotem pracy jest usprawnienie
programowania z użyciem typów w Glasgow Haskell Compiler.

Wśród popularnych
\js{Hm... ale do popularnych statycznie typowanych języków zalicza się Java,
C\#,... Na pewno byłbym ostrożny z nazwaniem Haskella popularnym.}
, statycznie typowanych języków istnieje wyraźne rozróżnienie
między poziomem typów, a poziomem termów.
\js{Nie jestem pewien czy użyłbym po polsku określenia ``poziom termów'' -
  obawiam się że słowo ``term'' może mieć w polskiej terminologii znaczenie
  typowo matematyczne, nie programistyczne.  Może lepiej mówić o wyrażeniach
  (ang. ``expressions'')?  I na pewno należy się tutaj kilka słów wyjaśniania.
  Pan i ja wiemy co to jest poziom typów i programowanie na poziomie typów -
  inni czytelnicy (recenzent...) prawdopodobnie nie. }
Zawarte w temacie pracy mechanizmy
programowania z użyciem typów służą zatarciu tej granicy.
\js{To ostatnie zdanie jest IMO nieprawdziwe - to co znajduje się w tej pracy na
  pewno nie służy zatarciu granicy.}
Dają one programistom
możliwości wyrażania logiki na poziomie typów przypominające \js{gram.} te dostępne na
poziomie termów. Gdy zostaną użyte, sprawdzenie typów w trakcie kompilacji może
wymagać obliczeń w celu znalezienia typów, do których ewaluują się zapisane na
poziomie typów wyrażenia.
\js{Mam problem ze zrozumieniem wcześniejszego zdania.}
Zwiększa to precyzję, z jaką możliwe jest opisywanie
zachowania programu.

Praca została poświęcona językowi Haskell i kompilatorowi Glasgow Haskell
Compiler. Wybór ten został podyktowany dostępnością rozszerzeń umożliwiających
programowanie z użyciem typów. GHC oferuje między innymi uogólnione algebraiczne
typy danych, promocję typów danych, zależności funkcyjne między parametrami w
klasach typów i rodziny typów.
\js{Warto tutaj przy każdym rozszerzeniu podać w nawiasie oryginalną, angielską
  nazwę.  Proszę pamiętać, że polska terminologia praktycznie nie istnieje i
  tworzy ją Pan tutaj samodzielnie. Warto tez podać odwołania do artykułów
  opisujących te rozszerzenia}
Dostępne są również pomocne biblioteki,
dostarczające na przykład typ danych Proxy
\js{Proszę formatować fragmentu kodu czcionką o stałej szerokości. Ponadto,
  Proxy jest części biblioteki base, więc w sumie nie pisałbym o pomocnej
  bibliotece, skoro jest to w standardzie.}
lub pozwalające na promocję zwykłych
funkcji do rodzin typów. Szczególnie istotnym rozszerzeniem są rodziny typów,
gdyż umożliwiają definiowanie na poziomie typów funkcji wykorzystujących
dopasowywanie wzorców oraz zależności funkcyjne
\js{chyba chodzi o klasy typów z zaleznościami funkcyjnymi?  }
, które dają równoważne możliwości, lecz wymagają stosowania stylu relacyjnego zamiast
funkcyjnego. Razem zapewniają możliwości zbliżone do tych, które mają języki z
typami zależnymi.

Rodziny typów zostały wprowadzone do Haskella w wersji 6.10.1, czyli w roku
2008. Do tej pory są przedmiotem prac naukowych, a ich implementacja jest
nieustannie rozwijana. Na przykład w 2015 roku wprowadzone zostały do GHC
różnowartościowe rodziny typów. Aktywny rozwój w połączeniu z faktem, iż type
checker to największy komponent GHC\cite{AOSA} skutkują tym, iż liczba
usprawnień oczekujących na wykonanie jest bardzo duża.
\js{Tą drugą część zdania trzeba jakoś doprecyzować, żeby było wiadomo że chodzi
  o usprawnienia dzięki którym programowanie na poziomie typów będzie
  wygodniejsze.}

Programowanie z użyciem typów możliwe jest w innych językach. Na przykład, Scala
oferuje do tego \foreign{path dependent types}, a w C++ możliwe jest
metaprogramowanie z użyciem szablonów. Istnieją też języki z typami zależnymi
jak Agda lub Idris. Wiele z tych języków ma implementacje o otwartych źródłach i
mogłoby służyć do realizacji tematu pracy zamiast Haskella.

\js{Mam poczucie, że ostatni akapit nic nie wniósł. }

Istnieje kilka implementacji języka Haskell. Spośród nich tylko dwie są zgode z
aktualną specyfikacją języka Haskell 2010, Glasgow Haskell Compiler i Utrecht
Haskell Compiler\cite{WikiImplementations}. Porównanie repozytoriów pozwala
stwierdzić, że spośród nich GHC jest bardziej rozwinięty. Różnica ta objawia się
w tym, iż rozszerzenia kompilatora pozwalające na programowanie na poziomie
typów jak rodziny typów istnieją wyłącznie w GHC\cite{UHCUserGuide}. Z tego
powodu w pracy został wybrany ten kompilator. \js{Wcześniej uzasadniał Pan wybór
kompilatora, teraz wraca Pan znowu do tego uzsadanienia.  Powinno być raz,
a dobrze.}

\todo[disable,inline]{Wstęp rozprawy powinien jasno określać tematykę i~zakres
  podejmowanego problemu.  Należy wskazać dlaczego dana tematyka została
  podjęta. Czy rozwiązania istniejące w~danej dziedzinie nie są wystarczające?
  Czy problem można rozwiązać inaczej? Czy podejmowany problem jest aktywnym
  tematem badawczym? Przed jakimi wyzwaniami stoi osoba podejmująca tematykę? Na
  tym etapie należy zarysować problem w~sposób ogólny.}

\section{Cele pracy}\label{sec:cele_pracy}

\js{W tym paragrafie jest nieco chaosu, dyskusja nie wydaje się przebiegać w
 sposób płynny.  Na pewno już na początku napisałbym czemu zależy nam na tych
 usprawnieniach.}

Celem pracy jest dokonanie jak największej liczby usprawnień w kompilatorze
GHC. Do organizacji pracy nad kompilatorem wykorzystywany jest system
Trac. Przez usprawnienie rozumiane są zmiany w GHC lub w powiązanych z tym
projektem bibliotekach, które zostały zgłoszone w systemie Trac. Zatem
usprawnienia, tak jak zgłoszenia, mogą polegać na naprawieniu błędu w działaniu
kompilatora, zaimplementowaniu nowej funkcjonalności lub wykonaniu pewnego
zadania.

W tej pracy podjęte zostały wyłącznie usprawnienia polegające na zmianach w
kodzie, w częściach związanych z rozszerzeniami opisanymi powyżej. Wybrane
usprawnienia to:

\begin{itemize}
 \item Ujednolicenie wyświetlania rodzin typów w błędach i ostrzeżeniach kompilatora.
 \item Dodanie ostrzeżeń o nieużywanych zmiennych w rodzinach typów.
 \item Naprawienie błędu z niewłaściwym parsowaniem zmiennych zaczynających się od podkreślnika w przypadku kompilacji z rozszerzeniem \code{NamedWildCards}.
\end{itemize}

Poprawne wprowadzenie zmiany w kodzie wymaga przejścia przez procedurę ustaloną
przez osoby mające prawa zapisu do repozytorium\cite{WikiFixingBugs}. Zgodnie z
nią wykonanie usprawnienia wymaga:

\begin{itemize}
  \item Informowania o stanie pracy w systemie Trac. Oznaczenie swojego konta jako właściciela zadania, uzupełnienie informacji na przykład o testach i powiązanych zadaniach, a na koniec oznaczenie zgłoszenia jako zrealizowanego.
  \item Przygotowania testów pokrywających wprowadzone zmiany. Dodanie ich do zestawu testów wykonywanych w czasie walidacji.
  \item Dokonanie zmian w kodzie w zgodzie z obowiązującymi konwencjami i z dokumentacją. Sporządzenie z nich commitów w systemie kontroli wersji git.
  \item Wysłanie łatki do systemu Phabricator, gdzie przejdzie ona inspekcję. Jeżeli nie będzie ku temu przeszkód, zmiany zostaną przeniesione do głównego repozytorium.
\end{itemize}

\section{Przegląd literatury}\label{sec:przeglad_literatury}

Podstaw teorii budowy kompilatorów dostarcza \cite{Dragon}. Autorzy opisują
budowę kompilatora, a następnie każdą fazę kompilacji. Ma to zastosowanie także
w przypadku kompilatora GHC. Krótki opis poświęcony konkretnie GHC dostępny jest
w \cite{AOSA}. Podstawy teorii typów zawarte są w \cite{TAPL}. Omówiony jest w
tej pozycji nietypowany i typowany rachunek lambda i kilka jego wzbogaconych
wersji. To bardzo istotne, gdyż pojęcia z rachunku lambda występują powszechnie
w dokumentacji, w kodzie GHC i są powszechnie używane w dyskusjach przez
programistów. Z drugiej strony część poświęcona podtypowaniu nie ma
bezpośredniego zastosowania do GHC oraz opis nie obejmuje typów
zależnych. Szczególnie warte uwagi w ramach wprowadzenia teoretycznego są
również dwa kursy, dostępne za darmo na platformie Coursera: \foreign{Automata}
prowadzony przez Jeffa Ullmana oraz \foreign{Programming Languages} prowadzony
przez Dana Grossmana. Są to zaadoptowane do formy kursu internetowego kursy
akademickie.

Praca nad GHC wymaga znajomości Haskella, w którym jest tworzony
kompilator. \cite{LearnYouAHaskell} i \cite{RealWorldHaskell} stanowią dobre
wprowadzenie do tego języka, obie pozycje są dostępne za darmo w Internecie. Z
rozszerzeniami zmienionymi w tej pracy, \code{TypeFamilies} i
\code{PartialTypeSignatures} można zapoznać się w poradniku użytkownika GHC, w
zasobach \cite{GuideTypeFamilies} i \cite{GuidePartialTypeSignatures}.

Strona systemu Trac projektu GHC zawiera wiele zasobów pozwalających
zainteresowanym programistom wdrożyć się w projekt. Jest tam między innymi opis
jak pobrać i skompilować źródła GHC\cite{WikiNewcomers}, opis procedury
wprowadzania zmian w kodzie\cite{WikiFixingBugs} oraz instrukcja konfiguracji i
wykorzystania narzędzia Phabricator\cite{WikiPhabricator}.

\todo[disable,inline]{W tym podrozdziale należy szczegółowo uzasadnić dlaczego wybrany został taki
a~nie inny temat pracy. Trzeba przede wszystkim zaprezentować aktualny stan
wiedzy w~danej dziedzinie. Oznacza to konieczność omówienia książek
(ew. artykułów naukowych bądź dokumentacji technicznej) z~których będzie się
korzystać w~trakcie rozprawy. Następnie należy wskazać -- tym razem już
konkretnie -- co nowego zamierza się zrobić. Podstawowymi celami tego
podrozdziału jest wprowadzenie czytelnika w~aktualny stand danej dziedziny
i~przekonanie go że \textbf{naprawdę warto zajmować się podjętym tematem}.}

\section{Układ pracy}\label{sec:uklad_pracy}

Rozdział \ref{chap:wstep} zawiera wprowadzenie i określenie tematu oraz celu
pracy. Rozdział \ref{chap:teoria} zawiera opis teorii budowy kompilatorów i
systemów typów potrzebnej związanej z tematem. Rozdział \ref{chap:technologie}
opisuje technologie i narzędzia wykorzystane w pracy. W rozdziale
\ref{chap:badania} przedstawiono opis dokonanych zmian w GHC. Rozdział
\ref{chap:podsumowanie} zawiera podsumowanie czy założone cele zostały
osiągnięte. Dodatek \ref{app:plyta} zawiera płytę CD z kodem aplikacji\todo{z
  patchami z moimi zmianami uzyskanymi git diffem?}
\js{Może być koopia repozytorium z Pańskimi gałęziami.}
i kopią tej pracy oraz
wykorzystanych źródeł internetowych.

\todo[disable,inline]{Tutaj należy zamieścić opis dalszej zawartości pracy.}
