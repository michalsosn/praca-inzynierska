\chapter{Wstęp}\label{chap:wstep}

Zakresem niniejszej pracy inżynierskiej jest informatyka, w szczególności
tworzenie dla języków funkcyjnych ze statycznymi systemami typów.
Przedmiotem pracy jest usprawnienie
programowania z użyciem typów w Glasgow Haskell Compiler.

W popularnych, statycznie typowanych językach programowania, gramatyka przewiduje
kilka rodzajów wyrażeń. Są to termy, służące do wyrażania obliczeń, które mają zostać
wykonane przez program po uruchomianiu i typy, które wiąże się z termami, by
rozróżnić termy poprawne oraz niepoprawne, których wykonanie do końca nie byłoby możliwe.
Niektóre języki zawierają jeszcze rodzaje (\foreign{kinds}), które pełnią wobec
typów taką rolę, jak typy wobec termów. Zawarte w temacie pracy mechanizmy programowania
z użyciem typów pozwalają na dodanie do typów, obok ich pierwotnej roli, możliwości
wyrażania obliczeń takich, jakie zwykle zarezerwowane są dla termów.
Obliczenia te wykonywane są na etapie sprawdzania typów w czasie kompilacji.

% \js{Nie jestem pewien czy użyłbym po polsku określenia ``poziom termów'' -
%   obawiam się że słowo ``term'' może mieć w polskiej terminologii znaczenie
%   typowo matematyczne, nie programistyczne.  Może lepiej mówić o wyrażeniach
%   (ang. ``expressions'')?  I na pewno należy się tutaj kilka słów wyjaśniania.
%   Pan i ja wiemy co to jest poziom typów i programowanie na poziomie typów -
%   inni czytelnicy (recenzent...) prawdopodobnie nie. }
% Starałem się stosować nazwy podążając za tą wzmianką w \cite{TAPL}
% "...we will use expression for all sorts of syntactic phrases
% (including term expressions, type expressions, kind expressions),
% reserving term for the more specialized sense of phrases representing computation."

Praca została poświęcona językowi Haskell i kompilatorowi Glasgow Haskell
Compiler. Wybór ten został podyktowany dostępnością rozszerzeń umożliwiających
programowanie z użyciem typów. GHC oferuje między innymi uogólnione algebraiczne
typy danych (\foreign{GADTs}), promocję typów danych (\foreign{data type promotion}),
zależności funkcyjne między parametrami w klasach typów
(\foreign(functional dependencies in multi-parameter type classes))
i rodziny typów (\foreign{Type families}).
\js{Warto tez podać odwołania do artykułów opisujących te rozszerzenia}
Biblioteka standardowa GHC dostarcza również pomocnych modułów, jak \code{Data.Type},
\code{Data.Proxy} i \code{GHC.TypeLits} z przydatnymi typami i funkcjami na poziomie
typów jak operatory logiczne.
Szczególnie istotnym rozszerzeniem są rodziny typów, gdyż umożliwiają
definiowanie na poziomie typów funkcji wykorzystujących
dopasowywanie wzorców oraz zależności funkcyjne w klasach typów,
które dają równoważne możliwości, lecz wymagają programowania w innym stylu,
relacyjnym zamiast funkcyjnego.

Programowanie z użyciem typów możliwe jest również w innych językach. Na przykład,
Scala oferuje do tego \foreign{path dependent types}, a w C++ możliwe jest
metaprogramowanie z użyciem szablonów. Istnieją też języki z typami zależnymi
jak Agda lub Idris. Wiele z tych języków ma implementacje o otwartych źródłach i
mogłoby służyć do realizacji pracy o takim samym zakresie zamiast Haskella.
% \js{Mam poczucie, że ostatni akapit nic nie wniósł.}
% Być może przy takim układzie paragrafów wygląda to lepiej?

Istnieje kilka implementacji języka Haskell. Spośród nich tylko dwie są zgodne z
aktualną specyfikacją języka Haskell 2010, Glasgow Haskell Compiler i Utrecht
Haskell Compiler\cite{WikiImplementations}. Porównanie repozytoriów pozwala
stwierdzić, że spośród nich GHC jest bardziej rozwinięty. Różnica ta objawia się
w tym, iż rozszerzenia kompilatora pozwalające na programowanie na poziomie
typów jak rodziny typów istnieją wyłącznie w GHC\cite{UHCUserGuide}. Z tego
powodu w pracy został wybrany ten kompilator.

Rodziny typów zostały wprowadzone do Haskella w wersji 6.10.1, czyli w roku
2008\cite{WikiIndexedTypes}. Do tej pory są przedmiotem prac naukowych, a ich implementacja jest
nieustannie rozwijana. Na przykład, w roku 2015 zostały wprowadzone do GHC
różnowartościowe rodziny typów. Aktywny rozwój świadczy o tym, iż jest to
temat bardzo aktualny. Czynnik ten sprawia też, iż wiele usprawnień do
nowo dodanych funkcji, ułatwiających korzystanie z nich, wciąż czeka na realizację.
Istotny jest też fakt, iż type checker to największy komponent GHC\cite{AOSA},
więc podczas modyfikacji z dużym prawdopodobieństwem powstają nowe błędy.
Podjęcie się realizacji tych usprawnień i naprawienia błędów są zatem dobrym tematem pracy.

\section{Cele pracy}\label{sec:cele_pracy}

Celem pracy jest dokonanie jak największej liczby usprawnień w kompilatorze
GHC. Przez usprawnienia rozumiane są zmiany w GHC lub w powiązanych z tym
projektem bibliotekach. Zmiany w GHC zgłaszane są w systemie Trac, który
używany jest w projekcie do organizacji pracy programistów. Jako cele pracy
wybranych zostało kilka zgłoszeń spośród tych znajdujących się w systemie Trac.

Podjęte zostały wyłącznie usprawnienia polegające na zmianach w
kodzie, w częściach związanych z rozszerzeniami opisanymi powyżej. Wybrane
usprawnienia to:

\begin{itemize}
 \item Ujednolicenie wyświetlania rodzin typów w błędach i ostrzeżeniach kompilatora.
 \item Dodanie ostrzeżeń o nieużywanych zmiennych w rodzinach typów.
 \item Naprawienie błędu z niewłaściwym parsowaniem zmiennych zaczynających się od podkreślnika w przypadku kompilacji z rozszerzeniem \code{NamedWildCards}.
\end{itemize}

Poprawne wprowadzenie zmiany w kodzie wymaga przejścia przez procedurę ustaloną
przez osoby mające prawa zapisu do repozytorium\cite{WikiFixingBugs}. Konieczne
jest zatem pisanie kodu akceptowalnej jakości, zgodnego z ustalonymi konwencjami.
Zgodnie z procedurą, wykonanie usprawnienia wymaga:

\begin{itemize}
  \item Informowania o stanie pracy w systemie Trac. Oznaczenie swojego konta jako właściciela zadania, uzupełnienie informacji na przykład o testach i powiązanych zadaniach, a na koniec oznaczenie zgłoszenia jako zrealizowanego.
  \item Przygotowania testów pokrywających wprowadzone zmiany. Dodanie ich do zestawu testów wykonywanych w czasie walidacji.
  \item Dokonanie zmian w kodzie w zgodzie z obowiązującymi konwencjami i z dokumentacją. Sporządzenie z nich commitów w systemie kontroli wersji git.
  \item Wysłanie łatki do systemu Phabricator, gdzie przejdzie ona inspekcję. Jeżeli nie będzie ku temu przeszkód, zmiany zostaną przeniesione do głównego repozytorium.
\end{itemize}

\section{Przegląd literatury}\label{sec:przeglad_literatury}

Podstaw teorii budowy kompilatorów dostarcza \cite{Dragon}. Autorzy opisują
budowę kompilatora, a następnie każdą fazę kompilacji. Ma to zastosowanie także
w przypadku kompilatora GHC. Krótki opis poświęcony konkretnie GHC dostępny jest
w \cite{AOSA}. Podstawy teorii typów zawarte są w \cite{TAPL}. Omówiony jest w
tej pozycji nietypowany i typowany rachunek lambda i kilka jego wzbogaconych
wersji. To bardzo istotne, gdyż pojęcia z rachunku lambda występują powszechnie
w dokumentacji, w kodzie GHC i są powszechnie używane w dyskusjach przez
programistów. Z drugiej strony część poświęcona podtypowaniu nie ma
bezpośredniego zastosowania do GHC oraz opis nie obejmuje typów
zależnych. Szczególnie warte uwagi w ramach wprowadzenia teoretycznego są
również dwa kursy, dostępne za darmo na platformie Coursera: \foreign{Automata}
prowadzony przez Jeffa Ullmana oraz \foreign{Programming Languages} prowadzony
przez Dana Grossmana. Są to zaadoptowane do formy kursu internetowego kursy
akademickie.

Praca nad GHC wymaga znajomości Haskella, w którym jest tworzony
kompilator. \cite{LearnYouAHaskell} i \cite{RealWorldHaskell} stanowią dobre
wprowadzenie do tego języka, obie pozycje są dostępne za darmo w Internecie. Z
rozszerzeniami zmienionymi w tej pracy, \code{TypeFamilies} i
\code{PartialTypeSignatures} można zapoznać się w poradniku użytkownika GHC, w
zasobach \cite{GuideTypeFamilies} i \cite{GuidePartialTypeSignatures}.

Strona systemu Trac projektu GHC zawiera wiele zasobów pozwalających
zainteresowanym programistom wdrożyć się w projekt. Jest tam między innymi opis
jak pobrać i skompilować źródła GHC\cite{WikiNewcomers}, opis procedury
wprowadzania zmian w kodzie\cite{WikiFixingBugs} oraz instrukcja konfiguracji i
wykorzystania narzędzia Phabricator\cite{WikiPhabricator}.

\todo[disable,inline]{W tym podrozdziale należy szczegółowo uzasadnić dlaczego wybrany został taki
a~nie inny temat pracy. Trzeba przede wszystkim zaprezentować aktualny stan
wiedzy w~danej dziedzinie. Oznacza to konieczność omówienia książek
(ew. artykułów naukowych bądź dokumentacji technicznej) z~których będzie się
korzystać w~trakcie rozprawy. Następnie należy wskazać -- tym razem już
konkretnie -- co nowego zamierza się zrobić. Podstawowymi celami tego
podrozdziału jest wprowadzenie czytelnika w~aktualny stand danej dziedziny
i~przekonanie go że \textbf{naprawdę warto zajmować się podjętym tematem}.}

\section{Układ pracy}\label{sec:uklad_pracy}

Rozdział \ref{chap:wstep} zawiera wprowadzenie i określenie tematu oraz celu
pracy. Rozdział \ref{chap:teoria} zawiera opis teorii budowy kompilatorów i
systemów typów potrzebnej związanej z tematem. Rozdział \ref{chap:technologie}
opisuje technologie i narzędzia wykorzystane w pracy. W rozdziale
\ref{chap:badania} przedstawiono opis dokonanych zmian w GHC. Rozdział
\ref{chap:podsumowanie} zawiera podsumowanie czy założone cele zostały
osiągnięte. Dodatek \ref{app:plyta} zawiera płytę CD z kodem aplikacji,
kopią tej pracy oraz wykorzystanych źródeł internetowych.
